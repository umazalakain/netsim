\documentclass{scrartcl}

% For mathematical symbols
\usepackage{amsmath}

% They are all over the place
\usepackage{todonotes}

% Appendices
\usepackage[header,title,titletoc]{appendix}
\renewcommand{\appendixname}{Appendix}

% License
\usepackage[
    type={CC},
    modifier={by-sa},
    version={3.0},
]{doclicense}

% Sane datestamps
\usepackage[yyyymmdd]{datetime}
\renewcommand{\dateseparator}{--}

% Haskell code highlighting
\usepackage{minted}
\newenvironment{code}{\VerbatimEnvironment\begin{minted}{haskell}}{\end{minted}}
\setminted{fontsize=\footnotesize,baselinestretch=1}

% Include images
\usepackage{graphics}

% Links and their colors
\usepackage[
  colorlinks=true,
  linkcolor=darkgray,
  citecolor=violet,
  urlcolor=darkgray,
  ]{hyperref}

% Change the geometry of some of the pages 
\usepackage{geometry}

% Put some pages in landscape
\usepackage{rotating}

\begin{document}

\title{Netsim}
\subtitle{Simulating distance-vector routing with synchronised periodic exchances}
\author{Uma Zalakain \\ \href{mailto:2423394z@student.gla.ac.uk}{2423394z@student.gla.ac.uk}}

\maketitle
\vfill
\doclicenseThis
\newpage

\section{Design overview}

\textit{Netsim} is a Haskell package made of a library containing all the logic
(\texttt{src/Language/Netsim.hs}) and an executable interface using that library
(\texttt{src/Main.hs}).

Instead of declaring new types, the library makes an extensive use of type
synonyms. Type synonyms are viewed as equivalent by the type-checker, but are
useful to convey information.

All information is contained within three types. All three types are indexed
first by a \textit{source} node and then by a \textit{destination} node: they
represent information regarding a relation $A \rightarrow B$.

\begin{description}
    \item [\texttt{Network}] constains the cost of every direct link from $A$ to
        $B$: static information about the network's toponomy. This information
        remains unchanged during periodic exchanges -- unless the user changes
        it to simulate broken links or cost changes.
        
        The function \texttt{bidir} is used to create undirected networks, where
        $A \rightarrow B \iff B \rightarrow A$. This is done by taking the union
        of the given network and its swapped copy -- where $A \rightarrow B
        \rightarrow cost$ gets transformed into $B \rightarrow A \rightarrow
        cost$.

    \item [\texttt{State}] contains the cost of the best route from $A$ to $B$
        together with the gateway for such route: all the runtime state. The
        gateway of a node must be directly linked to that node. 

    \item [\texttt{Ads}] contains advertisment messages sent from $A$ to $B$.
        These messages the periodic exchanges in the network. They contain
        distance vectors -- $destination \rightarrow cost$ maps.
\end{description}

On each \textit{tick} of the network, nodes first create advertisements
(containing a source, a destination and a distance vector) and then update their
tables with advertisements sent to them. Distance vectors created as part of
route advertisement messages can have split-horizon toggled.

All features related to execution are defined in \texttt{src/Main.hs}. First,
the user declares networks as files through a simple syntax. These networks are
then parsed. If the \texttt{-d} switch is present, they will be parsed as
directed networks, otherwise links will be made bidirectional. After this, the
user is presented with a simple shell where the main functionality is offered.

\section{User manual}
\subsection{Building}
\subsection{Installing}
\subsection{Running}

\section{Example walkthrough}

\section{Installation instructions}

\end{document}
