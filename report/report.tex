\documentclass{scrartcl}

% For mathematical symbols
\usepackage{amsmath}

% They are all over the place
\usepackage{todonotes}

% License
\usepackage[
    type={CC},
    modifier={by-sa},
    version={3.0},
]{doclicense}

% Sane datestamps
\usepackage[yyyymmdd]{datetime}
\renewcommand{\dateseparator}{--}

% Links and their colors
\usepackage[
  colorlinks=true,
  linkcolor=darkgray,
  citecolor=violet,
  urlcolor=darkgray,
  ]{hyperref}

\usepackage{tikz}

\begin{document}

\title{Netsim}
\subtitle{Simulating distance-vector routing with synchronised periodic exchances}
\author{Uma Zalakain \\ \href{mailto:2423394z@student.gla.ac.uk}{2423394z@student.gla.ac.uk}}

\maketitle
\vfill
\doclicenseThis
\newpage

\section{Design overview}

\textit{Netsim} is a Haskell package made of a library containing all the logic
(\texttt{src/Language/Netsim.hs}) and an executable interface using that library
(\texttt{src/Main.hs}).

Instead of declaring new types, the library makes an extensive use of type
synonyms. Type synonyms are viewed as equivalent by the type-checker, but are
useful to convey information.

All information is contained within three types. All three types are indexed
first by a \textit{source} node and then by a \textit{destination} node: they
represent information regarding a relation $A \rightarrow B$.

\begin{description}
    \item [\texttt{Network}] constains the cost of every direct link from $A$ to
        $B$: static information about the network's toponomy. This information
        remains unchanged during periodic exchanges -- unless the user changes
        it to simulate broken links or cost changes.
        
        The function \texttt{bidir} is used to create undirected networks, where
        $A \rightarrow B \iff B \rightarrow A$. This is done by taking the union
        of the given network and its swapped copy -- where $A \rightarrow B
        \rightarrow cost$ gets transformed into $B \rightarrow A \rightarrow
        cost$.

    \item [\texttt{State}] contains the cost of the best route from $A$ to $B$
        together with the gateway for such route: all the runtime state. The
        gateway of a node must be directly linked to that node. 

    \item [\texttt{Ads}] contains advertisment messages sent from $A$ to $B$.
        These messages the periodic exchanges in the network. They contain
        distance vectors -- $destination \rightarrow cost$ maps.
\end{description}

On each \textit{tick} of the network, nodes first create advertisements
(containing a source, a destination and a distance vector) and then update their
tables with advertisements sent to them. Stale broken links are cleared from
routing tables. Distance vectors created as part of route advertisement messages
can have split-horizon toggled.

All features related to execution are found in \texttt{src/Main.hs}. Networks
are declared by the user through a simple syntax and then parsed. If the
\texttt{-d} switch is present, links will be considered directed, otherwise
they will be made bidirectional. Finally, the user is presented with a simple
shell where the main functionality is offered: running the simulation, querying
the state, modifying the underlying network, activating split-horizon.

\section{User manual}

\textbf{Note}: an executable capable of running on Windows 10 x64 could not be
provided, as cross-compilation in Haskell can be tricky and the author does not
have a Windows 10 x64 machine.

This project is written in Haskell and uses Cabal to build itself. Both can be
installed by following the steps found at
\url{https://www.haskell.org/platform/windows.html}. 

\subsection{Compilation}

Assuming a shell with the project's directory as the current directory, the
project can be built as follows:

\begin{verbatim}
    # Updates the list of packages available online
    cabal update
    # Installs dependencies
    cabal install --dependencies-only
    # Builds the project
    cabal build
\end{verbatim}

The resulting binary will be found in \texttt{dist/build/netsim/netsim}.

\subsection{Installation}

Assuming a shell with the project's directory as the current directory, the
project can be installed as follows:

\begin{verbatim}
    # Updates the list of packages available online
    # Only necessary the first time
    cabal update
    # Install the project and its dependencies
    cabal install
\end{verbatim}

The generated output will specify the location of the generated binary. The
shell's path list should point to the directory in which the binary resides, in
which case it will be executable by running \texttt{netsim}.

\subsection{Execution}

Executing netsim without any arguments will output a brief help text. Netsim
expects a network file as its first argument. Unless netsim is ran without the
\texttt{-d} switch, network links are undirected.

Network files must have one link declaration per line (should support Windows'
carriage returns). Let \texttt{<SOURCE>} and \texttt{<DESTINATION>} be arbitrary
identifiers with no space in them, and let \texttt{<COST>} be an integer, then a
link is declared as follows:

\begin{verbatim}
<SOURCE> <DESTINATION> <COST>
\end{verbatim}

Once a network file is successfully parsed, netsim will present an interactive
shell. Unrecognised commands will result in help text being printed. The list of
commands follows:

\section{Example walkthrough}

\subsection{Normal convergence}

The figure below shows a network in which the direct link $A - B$ is more
costly than the route $A - D - C - B$. The file describing the network can be
loaded with \texttt{netsim examples/ex5.netsim}.

\begin{figure}[h!]
\centering
\begin{tikzpicture}[auto, thick, node distance=2cm]
    \node [draw,circle]            (A) {A};
    \node [draw,circle,below of=A] (B) {B};
    \node [draw,circle,right of=B] (C) {C};
    \node [draw,circle,above of=C] (D) {D};

    \draw (A) -- node {8} (B);
    \draw (B) -- node {3} (C);
    \draw (C) -- node {1} (D);
    \draw (D) -- node {2} (A);
\end{tikzpicture}
\end{figure}

Initially, each node only knows about its neighbours:
\begin{verbatim}
> table A
<D, 2, D>
<B, 8, B>
\end{verbatim}

After the first exchange, each node knows about nodes 2 hops away. The route
$A - B$ is still suboptimal:

\begin{verbatim}
> tick
> table A
<D, 2, D>
<B, 8, B>
<C, 3, D>
\end{verbatim}

After another exchange, each node knows about nodes 3 hops away, and thus $A$
knows about the optimal route to $B$:

\begin{verbatim}
> tick
> table A
<D, 2, D>
<B, 6, D>
<C, 3, D>
\end{verbatim}

\subsection{Slow convergence}

The figure below shows a network in which $A$ and $C$ need to talk through $B$.
The file describing the network can be loaded with \texttt{netsim
examples/ex4.netsim}.

After the initial exchanges that make the network stabilise, the link between
$A$ and $B$ will be broken. Depending on whether split-horizon is activated or
not, the subsequent exchanges might enter a loop and the network might never
stabilise.

\begin{figure}[h!]
\begin{tikzpicture}[auto, thick, node distance=2cm]
    \node [draw,circle]            (A) {A};
    \node [draw,circle,right of=A] (B) {B};
    \node [draw,circle,right of=B] (C) {C};

    \draw (A) -- node {1} (B);
    \draw (B) -- node {1} (C);
\end{tikzpicture}
\end{figure}

\subsubsection{Without split-horizon}

First, periodic exchanges occur until the network is stabilised:

\begin{verbatim}
> run
> table A
<B, 1, B>
<C, 2, B>
\end{verbatim}

Then the link between $A$ and $B$ is broken in both directions:

\begin{verbatim}
> fail A B
> fail B A
\end{verbatim}

When the periodic exchanges are resumed, $C$ tells $B$ that it has a route to
$A$ through $B$, and $B$ tells $C$ that it has a route to $A$ through $C$. This
exchanges go on in a loop, and the cost to $A$ keeps forever increasing.

\begin{verbatim}
> run
> table A
^C
\end{verbatim}

\textbf{Note}: Haskell is a lazy language where computation only happens when
the resulting data is needed. If evaluation was strict, the \texttt{run} command
would enter a loop. But because the resulting information is not needed until
the routing table for $A$ is displayed, the looping occurs on the \texttt{table
A} command.

Defining a low \textit{infinity} would prevent the looping behaviour, which
would terminate as soon as the cost of the route to $A$ reaches this number.

\subsubsection{With split-horizon}

As in the previous case, periodic exchanges are ran until the network is
stabilised and then the link between $A$ and $B$ is broken:

\begin{verbatim}
> run
> fail A B
> fail B A
\end{verbatim}

Before resuming the periodic exchange, the split-horizon feature is activated:

\begin{verbatim}
> split-horizon
\end{verbatim}

This prevents $C$ from sending to $B$ routes that have $B$ as gateway.
Likewise, it prevents $B$ from sending to $C$ routes that have $C$ as gateway.
The looping behaviour is therefore avoided and $A$ is successfully purged from
the routing tables:

\begin{verbatim}
> run
> table A

> table B
<C, 1, C>

> table C
<B, 1, B>
\end{verbatim}

\section{Status report}

\end{document}
